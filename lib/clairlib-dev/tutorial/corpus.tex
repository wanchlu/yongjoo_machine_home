\section{Corpus Representation in Clairlib}

In section 8.1 we presented the various ways of generating a Clairlib corpus from a data set. In this section, we describe Clairlib representation of corpora; i.e we describe the directory structure of the \emph{produced} directory generated by the command introduced in section 8.1.

\subsection{Directory Structure of a Corpus in Clairlib}

After running any of the commands that build a corpus from a data set, the following directory structure is built: (Assuming that the user chose the name \emph{produced} for the corpus base directory and \emph{MyCorpus} for the corpus name)
\\
\\
\begin{boxedverbatim}
produced (1)
|-- corpra (2)
    |-- MyCorpus (3)
        |-- 000
               |-- 00
        |-- 001

|-- Corpus-data (4)
    |-- MyCorpus (5)
    |-- MyCorpus-tf (6)
    |-- MyCorpus-tf-s (7)

|-- download (8)
    |-- MyCorpus (9)

|--MyCorpus.download.uniq (10)
\end{boxedverbatim}
\\
\\
In the listing above, the numbers between parentheses are added to the list just to make it easier to reference the corresponding files and directories. The details of this structure is (using the numbers in the list as references):

(1) The base directory of the corpora. One or more corpora can reside in the same base directory.

(2) This directory contains a sub directory for each corpus. Each directory contains the original dataset documents merged and formatted in a standard way (described in the next point.) The name of each directory is the name of the corresponding corpus.

(3) This is the directory of the \emph{MyCorpus} corpus. It contains the documents of the original dataset merged and formatted as follows:

\begin{itemize}
  \item Each file is formatted the TREC standard format

  \begin{boxedverbatim}
    <DOC>
        <DOCNO> 000-00-000002 </DOCNO>
        <DOCHDR> {the file URL} </DOCHDR>
        {The file content}
    </DOC>
  \end{boxedverbatim}

  \item Each 200 document are put together in one file. The files names are serial numbers from 00 to 39
  \item Each 40 file are put in one directory. The directories names are serial numbers starting from 000.
\end{itemize}

(4) This directory contains a sub directory for each corpus. Each directory contains the corpus metadata and indexing information (more details in the next point). The name of each directory is the name of the corresponding corpus.

(5) This is the data directory of the \emph{MyCorpus} corpus. It contains the following files

\begin{itemize}
  \item MyCorpus-docid-to-file: maps documents IDs to file names/
  \item MyCorpus-docid-to-url: maps documents IDs to their URLs.
  \item MyCorpus-url-to-docid: maps urls to documents IDs.
  \item MyCorpus-doclen: maps documents to their lengths.
  \item MyCorpus-idf: the inverse document frequency of the corpus terms.
  \item MyCorpus-idf-s: the stemmed Inverse Document Frequency of the corpus terms.
  \item MyCorpus.links: a list of all the hyperlinks within the corpus in (source, target) format.
  \item MyCorpus-tc: maps terms to their counts in the corpus.
  \item MyCorpus-tc-s: maps stemmed terms to their counts in the corpus.
\end{itemize}

(6) This directory contains the corpus term frequency. The terms are categorized in a 2 level alphabetical hierarchy; i.e terms are categorized on the first character, then the term in each category are categorized on the second character.

(7) The same as in (6) but with the terms stemmed.

(8) This directory contains a sub directory for each corpus. Each directory contains the original dataset copied as is. It preserves the same directory structure of the dataset.

(9) This directory contains the raw files of the \emph{MyCorpus} corpus as they came in the original dataset.

(10) This file maps the corpus files in \emph{./download} to their original location.
