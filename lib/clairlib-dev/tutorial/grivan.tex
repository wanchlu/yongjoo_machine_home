\subsection{Graph Partitioning Using the GirvanNewman Algorithm}

In this tutorial, you will use \emph{Clair::Network::GirvanNewman} to partition the Karate Club network into two partitions. For more information about Karate Club network, see \textbf{Spectral Partitioning Using Fiedler Vector} tutorial (the previous subsection).

\subsubsection{Read in the network file and create a Clair::Network object}

First you need to create a \emph{Clair::Network::Reader::GML} object
\\
\\
\begin{boxedverbatim}

 use Clair::Network::Reader::GML;
 my $reader=new Clair::Network::Reader::GML();

\end{boxedverbatim}
\\
\\
Then, pass the network filename to the \emph{read\_network} subroutine via the \emph{\$reader} object. This will return a \emph{Clair::Network} object.
\\
\\
\begin{boxedverbatim}
 use Clair::Network;
 my $filename = "karate.gml";
 my $net = $reader->read_network($filename);
\end{boxedverbatim}
\\
\subsubsection{Create a Clair::Network::GirvanNewman object}

Create a new \emph{Clair::Network::GirvanNewman} by calling its constructor
\\
\\
\begin{boxedverbatim}
 use Clair::Network::GirvanNewman;
 $GN = new Clair::Network::GirvanNewman($net);
\end{boxedverbatim}
\\
\subsubsection{Partition the graph}

To partition the graph, simply call \emph{partition()} subroutine via the \emph{\$GN} object. \emph{partition()} returns the result as a hash.
\\
\\
\begin{boxedverbatim}
 my $graphPartition = $GN->partition();
\end{boxedverbatim}
\\
\\
\$graphPartition is a hash with "node id" as key and "partition number" as value. The hierarchy structure for each node is represented as (0|1|2|1|...). So the number between "|" is the partition the node belongs to in a specific hierarchy.

You can use the dumper to print the contents of \emph{\$graphPartition}
\\
\\
\begin{boxedverbatim}
 use Data::Dumper
 print Dumper($graphPartition);
\end{boxedverbatim}
\\
\\
The output will be something like. This is what's called the dendrogram
\\
\\
\begin{boxedverbatim}
 $VAR1 = {
          '33' => '0|1|1|2|3|3|4|4|7|8|8|9|10...',
          '32' => '0|1|1|2|1|1|2|2|5|6|6|7|8...',
          . . .
          . . .
          '19' => '0|1|1|2|3|3|4|4|7|8|8|9|...',
          '5' => '0|0|0|0|4|4|5|6|3|1|1|2|...'
         };
\end{boxedverbatim}

