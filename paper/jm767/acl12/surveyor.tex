% File acl2012.tex
%
% Contact: gdzhou@suda.edu.cn
%%
%% Based on the style files for ACL2008 by Joakim Nivre and Noah Smith
%% and that of ACL2010 by Jing-Shin Chang and Philipp Koehn


\documentclass[11pt]{article}
\usepackage{acl2012}
\usepackage{times}
\usepackage{latexsym}
\usepackage{amsmath}
\usepackage{multirow}
\usepackage{graphicx}
\usepackage{url}
\DeclareMathOperator*{\argmax}{arg\,max}
\setlength\titlebox{2.5cm}    % Expanding the titlebox

\newcommand{\ignore}[1]{}

\newcommand{\B}{\mathcal{B}} 

\title{Surveyor: Generating Scientific Historical Notes}


\ignore{
\author{Dragomir R. Radev \and  Vahed Qazvinian \and Amjad Abu-Jbara \and  Wanchen Lu \and  Aditya Tayade\\
  University of Michigan, Ann Arbor\\
  {\tt \{radev,vahed,amjbara,wanchlu,atayade\}@umich.edu} \\
}
}


\date{}

\begin{document}
\maketitle

\begin{abstract}
We describe Surveyor, a system that automatically generates surveys of
collections of papers that represent scientific domains. Surveyor
extracts the main contributions of the underlying articles by
using heterogeneous sources: source texts and citations.  Using
Surveyor, we generate summaries of 10 topics in Natural Language
Processing corresponding to 10 chapters from the Jurafsky and Martin
textbook \emph{Speech and Language Processing}~\cite{JurafskyM08}. We evaluate the generated
summaries by comparing them to end-of-chapter summaries and
historical notes from the textbook and show that using heterogeneous
sources results in higher Rouge scores than state-of-art
summarization systems.
\end{abstract}

%%%%%%%%%%%%%%%%%%%%%
\section{Introduction}

Semi-supervised learning on graphs is a direction that has drawn great
attentions from NLP researchers. %The general idea behind various semi-supervised
%graphical methods is constructing a graph can be automated, while 
An abstract framework that we have often seen in graph based NLP is constructing
a graph in which each vertex represents an instance which can
be a word, a concept, a sentence, an article etc. Edges
are connected and weighted according to some manually selected
function that defines the closeness or similarity of any two instances in the
context of the application. 
One possible usage of the graph is to predict the
property (label) of a node via comparing the distances of this node to other
nodes with known labels. For example, //add a citation here

Another information we can obtain from the graph is the importance of a node.
We see algorithms such as PageRank\cite{}, HITS\cite{}, and graph based
summarization systems\cite{} .




In this paper, we introduce a model that shares with many other graphical
methods the idea of encoding similarities between instances into a graphical
structure and learning from unlabeled instances, while presents its novelty in
the following two aspects. First, the graph is bipartite. Example nodes and
feature nodes form two subsets of the bipartite graph. We will show later this
can be equivalently converted to a graph consisting of only example nodes
assuming a special edge weight definition. This
assumption simplifies the graph construction by a factor of the number of example nodes. 
Second, we observe from experiment that when the labeled training size is very
small compared to the entire graph, performance of the model is unstable due to the
randomness in sampling training examples. We extend the model with an active
learning technique that intellegently chooses the most ``informative'' unlabeled
examples to learn. Such property is well appreciated in applications where
labeled examples are very expensive to obtain.  


%%%%%%%%%%%%%%%%%%%%%
\section{Related Work}
\subsection{Terminology Recognition}
\subsection{Relation Extraction}
\cite{hearst98}
\cite{Brin:1998:EPR:646543.696220}
\subsection{Automatic Ontology Population}


%%%%%%%%%%%%%%%%%%%%%
\section{Datasets}
\label{sec:data}


\ignore{
\subsection{767}
The first author of this paper organized a seminar on Advanced NLP and IR
in Winter 2006 and Winter 2010. As part of the seminar, the students
in the class took turns and presented surveys of specific topics in
NLP and IR, two or three each week. In addition to their presentation, 
the students had to write chapter-length surveys of their topic. 

\begin{table*}[htbp]
\begin{tabular}{|l|}
\hline
Sentiment and Polarity Extraction - Arzucan Ozgur \\
Science Maps - Matt Simmons \\
Document Similarity - Ben Montgomery \\
Graph Random walks for Clustering, Classification, and Ranking - Ahmed Hassan \\
Language Networks - Yang Liu \\
String Kernels and Tree Kernels - Pradeep Muthukrishnan: \\
Spectral graph-based methods for NLP - Mike Bommarito \\
Sentence Simplification / Compression - Amjad Abu-Jbara \\
Text as Representation, Structured Data from Unstructured Text - Terry Szymanski \\
Automatic Structured Feature Mining for Instance Classification - Shilpa Arora \\
Information Diffusion In Graphs - Vahed Qazvinian \\
Recent Ideas in Summarization - Arzucan Ozgur \\
Evaluation methods in NLP - Abe Gong \\
Web Search Log Mining - Ahmed Hassan \\
Mining Online Discussions - Xiao Wei \\
Bibliographic Text Mining - Matt Simmons \\
Financial Networks - Mike Bommarito \\
Query Expansion - Ben Montgomery \\
Topic Segmentation - Vaibhav Mallya \\
Combining link-based and text-based networks - Pradeep Muthukrishnan \\
Mincuts - Yang Liu \\
Multiple Sequence Alignment - Terry Szymanski \\
Computational Advertising - Vahed Qazvinian \\
Active learning - Abe Gong \\
\hline
\end{tabular}
\end{table*}
}

In this section, we first describe the ACL Anthology Network, which is used as the source dataset for generating system surveys. We then explain our gold standard preparation from the Jurafsky and Martin textbook.

\subsection{The ACL Anthology Network}
The ACL Anthology\footnote{http://www.aclweb.org/anthology-new/} includes all papers published by ACL and related organizations as 
well as the Computational Linguistics journal over a period of four decades. 
\newcite{radev&al2009} have further processed this Anthology to produce the the ACL Anthology Network (AAN)\footnote{http://clair.si.umich.edu/clair/anthology/}.  The AAN includes more than $16,000$ papers, each distinguished with a unique ACL ID, together with their full-texts, abstracts, and citation information. It also includes other valuable meta-data such as author affiliations, citation and collaboration networks, and various centrality measures~\cite{radev&al2009,Joseph&Radev07}.
In our experiments, we generate a set of automatic summaries using the papers in AAN.

\subsection{Gold Standard Preparation}
We use 2 sets of gold standards both extracted from the Jurafsky and Martin textbook\footnote{we use the shorthand ``JM book'' in the rest of this paper} \emph{Speech and Language Processing} ~\cite{JurafskyM08}: end-of-chapter summaries and the historical notes.

\subsubsection{End-of-chapter Summaries}
We were fortunate to obtain the end of chapter summaries in the JM book in text format. Each summary is generally a few paragraphs long and explains the main points discussed in the chapter. We will refer to these gold standards as {\bf chapter summaries}.

\subsubsection{Historical Notes}
\label{sec:hist}
We also use the  {\bf historical notes} at the end of each chapter in the JM book as the second set of gold standards.
Each historical note, corresponding to one chapter, is generally 1-2 pages long and summarizes the history, early developments and the state-of-art methods in each NLP topic.

In order to prepare this gold standard, we first scanned the historical notes of the chapters as well as the references in the JM book. Next, we used a commercial OCR tool to convert the scanned files to plain text.
We further processed the OCR output by removing end-of-line hyphens and fixing sentence fragments and line breaks.\footnote{Parsing the bibliographies from the OCR output is more challenging than historical notes because of the smaller fonts and frequent out-of-vocabulary words such as author names. However, OCR errors are tolerated in bibliographies since we use minimum edit distance to find the corresponding papers in AAN.}
Cleaning-up references included identifying entry boundaries and combining multiple lines corresponding to one entry. 
 
We used the extracted references and citations in each historical note to extract the set of papers that are cited by ~\cite{JurafskyM08} and are part of AAN. 
We use these papers as the seed source papers to generate automatic summaries. 
To extract the list of AAN papers that are cited in each historical note, we first map each reference in the JM book to an AAN paper. 
First, for each reference we represent it by a vector of metadata that consists of the author names, title (stop words removed), canonical name of the venue, and publication year. 
We then compare these vectors with with AAN metadata and find the closest match by computing the minimum edit distance of corresponding metadata vectors when the publication dates agree. 
Finally, we manually verify the output of the above procedure and correct mismatches. 
%If a mismatch is found during manual verification, we either immediately discard the item if it is a non-AAN citation, or look up the paper manually.

\ignore{
The JM book uses the same citation convention as the ACL Anthology: ``first author (\& second author) (et al.), year''.
Using the list of matches between JM references list and AAN papers, we extract the list of papers that are cited in each historical note and are part of the AAN. We map a citation to a bibliography item by comparing the publication year and up to 2 authors. We resolve possible ties by manually identifying the right cited paper. 
}





%%%%%%%%%%%%%%%%%%%%%
\section{Approach}

\subsection{Terminology Extraction}


\subsection{Relation Extraction}


\subsection{Concept Class Classification}


\subsection{Validation}

Uncertainties and controversies sometimes arise during the 

%%%%%%%%%%%%%%%%%%%%%
\section{Experiments}

\subsection{PP attachemnt dataset}


1) baseline: backoff method

2) tumbl, randomly pick labeled examples

3) active learning

Todo: describe baseline, describe experiment settings

plot accuracy of 3 methods with different training size
%We use the backoff method \cite{} as our baseline, which is one of the that
%doesn't require.


\subsection{Named Entity Classification}

1) describe task and dataset

2) describe baseline DLCoTrain

3) experiment with NEC data

4) experiment DLCoTrain and tumbl on ppattach set

5) analysis


\subsection{AAN Terminology Extraction}

Not sure about this, if time is limited, show only preliminary
results, no comparison w/ other methods.

%%%%%%%%%%%%%%%%%%%%%
\section{Conclusion and Future Work}
\label{sec:con}
In this paper we present a framework based on the HITS algorithm that employs heterogeneous information (i.e., citations and source texts) to generate surveys of scientific paradigms. Using both Rouge and nugget-based evaluations, we show that our proposed system, Surveyor, generates summaries that have higher quality than the state-of-the-art methods when compared with end of chapter summaries and historical notes in Jurafsky and Martin NLP textbook.

In our work, we have used Jurafksy and Martin's end of chapter summaries as the gold standard written by written experts. We believe that the area of text summarization, and especially summarizing scholarly work can benefit from a wide range of expert written summaries that are produced more naturally, outside the context of multi-document summarization experiments. Other examples of such a gold standard source include ``further reading'' sections in the leading Information Retrieval textbook~\cite{manning07}, or survey papers published occasionally in journals such as Computational Linguistics.

One of the authors of this paper organized an NLP seminar previously. As part of the seminar, the students in the class took turns to present surveys of specific topics in NLP and Information Retrieval (IR) and wrote chapter-length surveys of their topics. 
In future work, we plan to make use of the surveys written by NLP students as  gold standard in evaluations. Compared to the chapters from JM book, these topics are more  specific and close to the latest development in NLP and IR. Examples include  Sentiment and Polarity Extraction, Science Maps, Spectral graph-based methods for NLP,  Information Diffusion In Graphs, Financial Networks and Query Expansion. 

In current work, we are using the papers cited in each chapter of the JM textbook
as seed source papers (i.e. we assume that the set of seminal papers on each topic are known).  However in the science community, there are  thousands more papers that are related to a given topic.  In the future,  we will work on a method of automatically identifying the most influential papers that represent a specific topic from the vast range of publications.






\bibliographystyle{acl2012}
\bibliography{ref}

\end{document}
